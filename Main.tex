\documentclass[12pt,a4paper]{article}

\usepackage{amsmath,amssymb,amsthm}
\usepackage{geometry}
\usepackage{hyperref}
\usepackage{parskip}
\usepackage{titlesec}
\usepackage{enumitem}
\usepackage{microtype}
\usepackage{mathtools}
\usepackage{amsfonts}

\geometry{margin=1in}

\hypersetup{
    colorlinks=true,
    linkcolor=blue,
    citecolor=blue,
    urlcolor=blue,
    pdftitle={The Incompleteness of Observation},
    pdfauthor={Alex Maybaum}
}

\newtheoremstyle{named}{}{}{\itshape}{}{\bfseries}{.}{ }{}
\theoremstyle{named}
\newtheorem*{theorem*}{Theorem}
\newtheorem*{conjecture*}{Conjecture}

\title{\textbf{THE INCOMPLETENESS OF OBSERVATION}\\[0.5em]
\large Why Quantum Mechanics and General Relativity Cannot Be Unified From Within}
\author{Alex Maybaum}
\date{February 2026\\[0.3em]
\textit{Status: DRAFT PRE-PRINT}\\
\textit{Classification: Theoretical Physics / Foundations}}

\begin{document}

\maketitle
\thispagestyle{empty}

\begin{center}
\rule{\linewidth}{0.4pt}
\end{center}

\begin{abstract}
The incompatibility between quantum mechanics and general relativity is a structural consequence of embedded observation. Any observer that is part of the universe it measures must access reality through projections that discard inaccessible degrees of freedom defined by spacetime's causal structure. The \textbf{Observational Incompleteness Theorem} applies Wolpert's (2008) inference limits: quantum and gravitational vacuum energy measurements are variance-type and mean-type projections of a shared hidden sector, and no embedded device can simultaneously determine both. Under this identification, the $10^{122}$ cosmological constant discrepancy is reinterpreted as a measurement of $\sim 10^{244}$ hidden-sector degrees of freedom.

The \textbf{Trace-Out Conjecture} then demonstrates that, assuming only classical Liouville dynamics, classical general relativity, and classical probability theory, marginalizing over the correlated hidden sector produces generically indivisible stochastic dynamics on the visible sector. By Barandes' stochastic-quantum correspondence, this indivisible process is exactly equivalent to unitary quantum mechanics. The Schr\"{o}dinger equation emerges as the unique time-local description of history-dependent classical marginals. The framework yields falsifiable predictions, including post-merger gravitational wave echoes and a stochastic gravitational noise floor whose spectral density is anchored to the hidden-sector dimensionality.
\end{abstract}

\begin{center}
\rule{\linewidth}{0.4pt}
\end{center}

\tableofcontents
\newpage

%% ============================================================
\section{The Problem}

\subsection{The Incompatibility}
Quantum mechanics and general relativity are extraordinarily successful yet structurally incompatible. This paper shows that the incompatibility is not a defect requiring unification but a consequence of embedded observation.

\subsection{The Cosmological Discrepancy}
The sharpest manifestation of the QM--GR incompatibility is the \textbf{cosmological constant problem}. It concerns the single quantity that both frameworks predict: the energy density of empty space, $\rho_{\text{vac}}$.

\textbf{Quantum mechanics} computes the vacuum energy by summing zero-point fluctuations of all quantum field modes up to the Planck scale:
\[
\rho_{\text{QM}} \sim \frac{E_{\text{Pl}}^{\,4}}{(\hbar c)^3} \sim 10^{113} \; \text{J/m}^3
\]

\textbf{General relativity} measures the vacuum energy through its gravitational effect --- the accelerated expansion of the universe:
\[
\rho_{\text{grav}} = \frac{\Lambda \, c^2}{8\pi G} \sim 6 \times 10^{-10} \; \text{J/m}^3
\]

The ratio is conventionally rounded to $10^{120}$ in the literature \cite{Weinberg1989,Martin2012,Carroll2001}, or more precisely $10^{122}$. This paper offers a different interpretation: \textbf{neither calculation is wrong. They disagree because they are answering fundamentally different questions about the same thing}.

%% ============================================================
\section{Observational Incompleteness}

\subsection{Observers Are Embedded}
Wolpert (2008) \cite{Wolpert2008} proved that any physical device performing observation, prediction, or recollection --- an ``inference device'' --- faces fundamental limits on what it can know about the universe it inhabits. These limits hold \textbf{independent of the laws of physics}. They follow solely from the logical structure of a device that is embedded in the system it attempts to describe, forcing any observation to be a surjective, many-to-one mapping from the total state to the device's output.

\subsection{The Hidden Sector}
Let the full state space be partitioned into degrees of freedom accessible to observers (the \emph{visible sector}) and degrees of freedom that are not (the \emph{hidden sector}, denoted $\Phi$). The hidden sector consists of trans-horizon modes beyond the cosmological horizon, sub-Planckian degrees of freedom below the observer's resolution limit, and black hole interiors. The partition is defined entirely by classical general relativity. No quantum concept enters the definition of what is hidden from the observer.

\subsection{Two Projections of the Same Thing}
To make the connection to Wolpert's framework precise, the two vacuum-energy measurements are formalized as distinct inference targets. Let the hidden sector's microstate be characterized by a distribution over $N$ mode energies $\{E_i\}$. An embedded observer cannot access this distribution directly; any measurement is a surjective (many-to-one) mapping from the full microstate to a single output. Following Wolpert, two such inference targets are defined:

\textbf{Target 1: The Total Unsigned Power (QM's Variance-Type Projection).} QFT's operational access to the vacuum is through the fluctuation power spectrum. Because zero-point energies $\frac{1}{2}\hbar\omega$ are positive-definite, the QFT vacuum energy sums their absolute magnitudes:
\[
V = \sum_{i=1}^{N} |E_i|
\]
This is a variance-type target in the sense that it measures the total unsigned spread of the distribution --- it grows linearly with $N$ regardless of the signs of individual contributions, analogous to how the variance of a sum of independent variables grows linearly with the number of terms. Note that $V$ is not literally a statistical variance; the analogy is in the scaling behavior and in the fact that $V$ is insensitive to correlations among the signs of contributions.

\textbf{Target 2: Mean-Field Pressure (Gravity's Mean-Type Projection).} The Einstein field equations couple spacetime curvature to the macroscopic expectation value of the stress-energy tensor. The gravitational projection is sensitive to the \emph{net signed sum}:
\[
M = \langle T_{00} \rangle = \sum_{i=1}^{N} E_i
\]
where the $E_i$ now carry their physical signs. In a system with many degrees of freedom spanning both bosonic and fermionic sectors, broken symmetry phases, and the variety of interactions present in a realistic field theory, the individual contributions generically carry both positive and negative signs. \textbf{This is a substantive physical assumption}: the hidden sector does not possess an unbroken symmetry that would align all contributions with a common sign. Under this assumption, $M$ is a sum of quasi-independent signed terms, and the central limit theorem gives the characteristic scaling $|M| \sim \sqrt{N}$.

The two targets $V$ and $M$ are distinct surjective functions of the same underlying microstate. In Wolpert's framework, two inference targets defined on the same system are subject to joint constraints when the inference device is embedded in that system: the device's own state must be consistent with both measurements simultaneously, imposing a trade-off on joint accuracy. The specific product bound on the mean-squared errors of $V$ and $M$ depends on the coupling structure and remains open (Section \ref{sec:open}). What the framework establishes is the structural claim: these are complementary projections of a shared sector, and their disagreement is a signature of the projection rather than an error in either theory.

\subsection{The Observational Incompleteness Theorem}

\begin{theorem*}[Observational Incompleteness]
Let the universe be partitioned into visible and hidden sectors, and let the observer's projection from the full state to the visible sector be many-to-one. Define the variance-type target $V = \sum |E_i|$ and the mean-type target $M = \sum E_i$ on the hidden-sector distribution. Then:
\begin{enumerate}[label=(\roman*)]
    \item $V$ and $M$ are distinct surjective functions of the same microstate;
    \item by Wolpert's results \cite{Wolpert2008}, any embedded inference device faces fundamental limits on simultaneously determining both targets; and
    \item under the physical assumption that hidden-sector contributions carry quasi-random signs, $V \propto N$ while $|M| \propto \sqrt{N}$, producing a ratio $V/|M| \propto \sqrt{N}$.
\end{enumerate}
\end{theorem*}

\textbf{Status of the theorem.} Claim (i) is definitional. Claim (ii) follows from Wolpert's general results on embedded inference; the specific joint accuracy bound for these particular targets has not yet been derived, but the structural form of Wolpert's constraints is sufficient to establish the incompatibility. Claim (iii) rests on the physical assumption stated in Section 2.3 regarding the sign structure of hidden-sector contributions. Sharpening the quantitative bound remains a priority open problem (Section \ref{sec:open}).

\subsection{Extracting the Hidden-Sector Dimensionality}
The theorem immediately reframes the cosmological constant discrepancy as an informative physical quantity rather than a paradox. Since $V \propto N$ and $|M| \sim \sqrt{N}$, the ratio of the two projections directly encodes the hidden sector's dimensionality:
\[
\frac{V}{|M|} \sim \frac{N}{\sqrt{N}} = \sqrt{N}
\]
Setting this equal to the observed discrepancy:
\[
\sqrt{N} \sim 10^{122} \implies N \sim 10^{244}
\]
The $10^{122}$ ratio is thus a quantitative signature of observational incompleteness --- encoding the dimensionality of the sector that embedded observers cannot access.

%% ============================================================
\section{Classical Premises and Marginal Dynamics}

The Observational Incompleteness Theorem establishes that embedded observers face irreducible epistemic constraints when probing the hidden sector. The remainder of this paper pursues a considerably stronger claim: that quantum mechanics itself is a necessary consequence of these constraints. The argument's credibility rests entirely on the transparency of its premises and the rigor of each mathematical step. The derivation proceeds from three classical premises through a chain of established mathematical results, assuming no quantum postulate at any stage.

\begin{itemize}
    \item \textbf{Premise 1: Classical Statistical Dynamics.} The total universe is a classical statistical system evolving deterministically via the Liouville equation:
    \[
    \frac{\partial \rho}{\partial t} = \{H, \rho\}
    \]
    \item \textbf{Premise 2: Classical General Relativity as Causal Structure.} Einstein's field equations determine the causal structure of spacetime, creating the absolute information barriers that define the hidden sector.
    \item \textbf{Premise 3: Classical Probability Theory.} All statistical inference follows from Kolmogorov's axioms. Observational predictions are classical expectation values.
\end{itemize}

The observer's operational description of the visible sector is therefore a \textbf{marginal stochastic matrix}:
\[
T_{ij}(t, t_0) = \int d\gamma_h \; P(\gamma_h \mid v_i, t_0) \; \mathbf{1}\!\left[\Phi_t(v_i, \gamma_h) \in v_j\right]
\]
The central question: can $T$ be factored as a strictly divisible Markov process, or does marginalization over the hidden sector break divisibility?

%% ============================================================
\section{The Trace-Out Conjecture}

\subsection{Statement of the Conjecture}

\begin{conjecture*}[Trace-Out]
Let the universe be a classical statistical system governed by deterministic Liouville dynamics, with classical general relativity partitioning it into visible and hidden sectors ($N \sim 10^{244}$ inaccessible degrees of freedom). Shared causal history and exact conservation laws enforce non-factorizable joint distributions. The marginal stochastic dynamics on the visible sector is then generically indivisible with probability 1 within the GOE ensemble, extending by Wigner-Dyson universality to any sufficiently complex hidden-sector Hamiltonian. By Barandes' stochastic-quantum correspondence \cite{Barandes2025,Barandes2023} (adopted here as a premise) and the continuous complex-phase mapping, this indivisible classical process is exactly equivalent to unitary quantum mechanics. The quantum formalism is therefore the mandatory data-compression algorithm for any embedded observer forced to marginalize deterministic classical mechanics over a correlated hidden sector.
\end{conjecture*}

\subsection{Indivisibility via Random Matrix Theory}

The visible and hidden sectors are necessarily correlated: prior to cosmological horizon formation these degrees of freedom were in direct causal contact, and Noether conservation laws ($E_{\text{total}} = E_v + E_h = \text{const}$) rigidly couple them at all times. Because $\rho(\gamma_v, \gamma_h) \neq \rho_{\text{vis}}(\gamma_v)\,\rho_{\text{hid}}(\gamma_h)$, the marginal stochastic dynamics is generically non-divisible. Two established results support this qualitatively: Kemeny and Snell (1960) showed that coarse-graining retains the Markov property only under strong lumpability conditions, which Gurvits and Ledoux (2005) proved are nowhere dense; independently, Casanellas et al.\ (2023) proved that embeddable stochastic matrices form a proper semi-algebraic subset of strictly lower dimension for $n \geq 3$. The following argument elevates these transversality observations to a statistical proof.

The hidden Liouvillian $\mathcal{L}_{\text{hid}}$ is modeled as a large matrix drawn from the Gaussian Orthogonal Ensemble (GOE) --- the maximally agnostic model consistent with time-reversal symmetry (real matrix elements) and inaccessibly high dimensionality ($N \sim 10^{244}$). Writing the total Liouvillian as $\mathcal{L} = \mathcal{L}_0 + \delta \mathcal{L}_h$ and applying the Dyson series to the Mori-Zwanzig projected propagator, the first-order variation of the transition matrix is:
\[
\delta T_{ij}(t) = \int_0^t d\tau \, \langle j | P e^{\mathcal{L}_0(t-\tau)} \delta \mathcal{L}_h e^{\mathcal{L}_0\tau} P | i \rangle
\]

Conservation-law correlations prevent $e^{\mathcal{L}_0\tau}$ from factorizing, so the covariance of the transition elements is:
\[
C_{(ij)(kl)} = \langle \delta T_{ij} \delta T_{kl} \rangle \propto \mathrm{Tr}(\mathcal{L}_{\text{int}}^\dagger \mathrm{Cov}(\delta \mathcal{L}_h) \mathcal{L}_{\text{int}})
\]

In the GOE limit ($N \gg n$), the Jacobian of the marginalization map $\mathcal{M}: (H, \rho_0, t) \to T(t)$ has full row rank, so $C$ is strictly positive-definite and the induced measure $P(T)$ has full-dimensional support on the stochastic simplex $\mathcal{S}_n$. Since embeddable Markov matrices $\mathcal{E}_n$ occupy a zero-volume submanifold, marginalization yields strictly indivisible dynamics with probability 1. Extending the proof from GOE spectral statistics to the full Liouvillian structure remains open (Section \ref{sec:open}), though Wigner-Dyson universality provides strong physical grounds for the generalization.

Barandes (2025) \cite{Barandes2025,Barandes2023} established a bijection between indivisible stochastic processes on $n$ configurations and unitary quantum systems on $\mathcal{H}$, with $T_{ij}(t) = |U_{ij}(t)|^2$. This elevates the probability-1 indivisibility result to a direct equivalence: the marginal dynamics of the visible sector is exactly quantum mechanical.

\subsection{The Continuous Limit: Emergence of the Schr\"{o}dinger Equation}

The discrete equivalence must extend to continuous configuration spaces to recover the Schr\"{o}dinger equation. A visible particle coupled to the hidden sector obeys the non-Markovian generalized Langevin equation:
\[
m\ddot{q} + \nabla V(q) + \int_0^t K(t-\tau) \dot{q}(\tau) \, d\tau = F_{\text{fluct}}(t)
\]
Because the hidden sector is finite (bounded by the cosmological horizon), the memory kernel $K(t-\tau)$ retains a non-zero correlation time $\tau_E$, and the dynamics are strictly indivisible.

\textbf{Step 1: Necessity of a complex wave function.} A real density $P(q,t)$ carries one degree of freedom per spatial point, but the memory integral makes future evolution depend on the entire past trajectory --- the dynamics requires two local fields (the current distribution and its accumulated momentum history). This is a counting argument, not an approximation issue. Time-locality therefore forces the introduction of a conjugate phase field $S(q,t)$, with $\psi(q,t) = \sqrt{P(q,t)} \, e^{iS(q,t)/\hbar}$. The phase gradient encodes the integrated historical momentum:
\[
\nabla S(q,t) = m\dot{q} + \int_0^t K(t-\tau) \dot{q}(\tau) \, d\tau
\]
That is, $\nabla S$ is the minimal additional field needed to promote the non-Markovian integro-differential dynamics into an equivalent time-local PDE system. Identifying the phase as a consequence of non-Markovianity, rather than postulating it, is the central novel element of this derivation.

\textbf{Step 2: Hydrodynamic equations.} Conservation of probability and energy conservation enforced by the hidden-sector bath yield a continuity equation and a modified Hamilton-Jacobi equation with the quantum potential
\[
Q = -\frac{\hbar^2}{2m} \frac{\nabla^2 \sqrt{P}}{\sqrt{P}},
\]
where $\hbar$ enters as the scale factor set by the hidden sector's fluctuation-dissipation balance. Substituting $\psi = \sqrt{P} \, e^{iS/\hbar}$, both equations reduce to:
\[
i\hbar \frac{\partial \psi}{\partial t} = \left[ -\frac{\hbar^2}{2m}\nabla^2 + V(q) \right] \psi
\]
This is the Madelung--Bohm hydrodynamic reformulation (Madelung 1927, Bohm 1952) read in reverse --- from classical non-Markovian dynamics \emph{toward} the quantum formalism. The Schr\"{o}dinger equation is thereby the unique time-local PDE generating the indivisible marginal transition probabilities of a history-dependent classical system.

\textbf{The Wallstrom objection.} Wallstrom (1994) \cite{Wallstrom1994} showed that the Madelung equations are not equivalent to the Schr\"{o}dinger equation without the quantization condition $\oint \nabla S \cdot dl = 2\pi n \hbar$. In the present framework, this gap is addressed by the discrete stochastic foundation of Sections 4.1--4.2: the indivisible stochastic matrices carry the full quantum structure via Barandes' correspondence, including the discrete analogue of topological quantization (single-valuedness of $U_{ij}$). The continuous Schr\"{o}dinger equation is obtained as the continuum limit of this discrete dynamics, so phase quantization is inherited rather than postulated. For finite-dimensional configuration spaces this follows from continuity of the stochastic-to-unitary map; for infinite-dimensional field-theoretic spaces the continuum limit remains open (Section \ref{sec:open}). The Wallstrom objection is therefore substantially mitigated but not fully eliminated for field theories.

\subsection{Experimental Predictions}

The framework generates two falsifiable predictions anchored to the hidden-sector dimensionality $N \sim 10^{244}$.

\textbf{Prediction 1: Gravitational Wave Echoes.}
If spacetime curvature is a mean-field thermodynamic variable of the hidden sector, post-merger black hole ringdown should exhibit echoes reflecting the hidden sector's discrete granularity. For a fast scrambler with entropy $S$, the scrambling time scales as $t_{\text{scr}} \sim \beta \ln S$, where $\beta \sim r_s / c$ is the inverse Hawking temperature. Taking $S \sim \sqrt{N} \sim 10^{122}$ (consistent with the holographic bound on the observable universe):
\[
\Delta t_{\text{echo}} \sim \frac{r_s}{c} \ln(10^{122}) \sim \frac{r_s}{c} \times 281
\]
For a 30 $M_\odot$ post-merger remnant ($r_s \approx 90$ km), this yields $\Delta t_{\text{echo}} \sim 8 \times 10^{-5}$ s, in the $10^{-5}$--$10^{-4}$ s range accessible to current LIGO/Virgo/KAGRA post-merger analyses. The echo amplitude depends on the reflectivity of the effective hidden-sector boundary; a quantitative estimate requires solving the effective boundary conditions (Section \ref{sec:open}).

\textbf{Prediction 2: Stochastic Gravitational Noise Floor.}
Irreducible hidden-sector fluctuations must source a stochastic gravitational wave background. A naive dimensional estimate yields:
\[
\Omega_{\text{gw}}^{\text{naive}} \sim \frac{\rho_{\text{QM}}}{\rho_c} \times \frac{1}{\sqrt{N}} \sim \frac{10^{113}}{10^{-10}} \times 10^{-122} \sim 10^{1}
\]
This $\mathcal{O}(1)$ value would violate existing bounds from LIGO/Virgo, pulsar timing arrays, and Big Bang nucleosynthesis, indicating the estimate is an overestimate. Two suppression mechanisms account for the discrepancy: \textbf{spectral dilution} (the $N$ modes span $\sim$61 decades in frequency, so the spectral density at any given frequency is far below the integrated total) and \textbf{quadrupole coupling} (only modes with non-zero time-varying quadrupole moments source gravitational radiation, and sub-Planckian degrees of freedom may lack the spatial extent for efficient emission). The precise spectral shape $\Omega_{\text{gw}}(f)$ depends on the hidden sector's mode structure and is not yet determined.

This constitutes a genuine falsifiability test: the framework predicts a stochastic background must exist, while the amplitude is constrained from above by observation. If no physically reasonable mode structure can reconcile the spectral decomposition with observational upper bounds, the framework is falsified. Computing $\Omega_{\text{gw}}(f)$ across the MHz--GHz band (where astrophysical foregrounds are minimal) is a priority for follow-up work.

%% ============================================================
\section{Discussion and Open Problems}

\subsection{Logical Independence of the Central Claims}
The two central results of this paper are logically independent. The Observational Incompleteness Theorem depends only on Wolpert's inference limits, the causal structure of general relativity, and the identification of QM and GR vacuum energy measurements as variance-type and mean-type projections. It does not require the Trace-Out Conjecture or any claim about the emergence of quantum mechanics. Conversely, the Trace-Out Conjecture depends on classical Liouville dynamics, classical probability theory, the existence of a correlated hidden sector, and Barandes' stochastic-quantum correspondence --- but it does not require the specific identification of the cosmological constant discrepancy as a measurement of $N$. Each result should be evaluated on its own premises. Together, they form a mutually reinforcing picture: the Observational Incompleteness Theorem addresses \emph{why} embedded observers face complementary descriptions, while the Trace-Out Conjecture addresses \emph{what dynamical framework} those descriptions must take. The independent convergence with Wetterich's (2001--2025) incomplete-statistics route to quantum mechanics from classical subsystems provides additional corroboration.

\subsection{Open Problems}
\label{sec:open}

\begin{enumerate}
    \item \textbf{Continuous-Variable Extension (Fields).} Extend the phase-mapping framework to continuous infinite-dimensional phase spaces required for relativistic quantum field theory. This includes demonstrating that the discrete-to-continuous limit preserves the topological quantization conditions that address the Wallstrom objection.

    \item \textbf{Quantitative Bounds.} Determine the exact relationship between the dimensionality of the hidden sector $N$, the strength of correlations, and the degree of macroscopic indivisibility.

    \item \textbf{The Continuous Precision Trade-Off.} Derive the specific product bound on the mean-squared errors of the variance-type target $V$ and the mean-type target $M$ from Wolpert's framework, sharpening the Observational Incompleteness Theorem from a structural claim into a quantitative uncertainty relation.

    \item \textbf{Echo Amplitude and Spectral Shape.} Derive the effective reflectivity of the hidden-sector boundary from first principles to produce a quantitative amplitude prediction for gravitational wave echoes. Compute the spectral shape $\Omega_{\text{gw}}(f)$ of the stochastic noise floor across the MHz--GHz band, incorporating spectral dilution and quadrupole coupling suppression, and determine whether the resulting amplitude is consistent with existing observational upper bounds. This is the most pressing empirical test of the framework.

    \item \textbf{Universality Beyond GOE.} Extend the full-rank covariance result from the GOE ensemble to broader classes of Hamiltonians, establishing that the indivisibility conclusion does not depend on the specific random matrix model.

    \item \textbf{Sign Structure of the Hidden Sector.} Provide a first-principles argument --- or identify physical conditions under which --- the hidden-sector energy contributions carry quasi-random signs, as assumed in the CLT argument underlying the OIT.
\end{enumerate}

%% ============================================================
\section{Conclusion}

This paper has shown that the structural incompatibility between quantum mechanics and general relativity is not a failure of either theory, but a mathematical consequence of embedded observation within a classical universe. Applying Wolpert's inference limits to the causal horizon structure of general relativity, the \textbf{Observational Incompleteness Theorem} establishes that vacuum energy measurements are complementary projections of a shared hidden sector. Under the physical assumption that hidden-sector contributions carry quasi-random signs, the $10^{122}$ cosmological constant discrepancy encodes the hidden sector's dimensionality, yielding $N \sim 10^{244}$ degrees of freedom.

The \textbf{Trace-Out Conjecture} is derived through a chain of established mathematical results with clearly stated assumptions. Modeling the hidden-sector Liouvillian as a Gaussian Orthogonal Ensemble (GOE), the covariance matrix of marginal transition elements is full-rank, establishing that the observer's description of the visible sector is indivisible with probability 1. By Wigner-Dyson universality, this result extends to physically realistic Hamiltonians of sufficient complexity. The history-dependent memory kernel $K(\tau)$, which explicitly breaks classical divisibility, is dual to the complex phase gradient $\nabla S$ in a unitary representation.

Hilbert space, the Born rule, and the Schr\"{o}dinger equation emerge not as fundamental laws, but as the mandatory data-compression algorithms for any observer forced to trace out a correlated, causally inaccessible hidden sector. Quantum mechanics is the ``epistemic shadow'' cast by classical mechanics when viewed from behind a relativistic horizon --- the universe a deterministic system that appears quantum precisely because the observer is part of the system it attempts to measure. The remaining open problems identified in Section \ref{sec:open} --- particularly the quantitative Wolpert bound, the hidden-sector spectral structure, and the rigorous universality extension beyond GOE --- define the path toward complete closure.

%% ============================================================
\begin{thebibliography}{99}

\bibitem{Wolpert2008}
D. H. Wolpert, ``Physical Limits of Inference,'' \textit{Physica D} \textbf{237}, 1257--1281 (2008).

\bibitem{Weinberg1989}
S. Weinberg, ``The Cosmological Constant Problem,'' \textit{Rev. Mod. Phys.} \textbf{61}, 1 (1989).

\bibitem{Martin2012}
J. Martin, ``Everything You Always Wanted to Know About the Cosmological Constant Problem (But Were Afraid to Ask),'' \textit{C. R. Phys.} \textbf{13}, 566--665 (2012).

\bibitem{Ahmed2004}
M. Ahmed, S. Dodelson, P. B. Greene, and R. Sorkin, ``Everpresent $\Lambda$,'' \textit{Phys. Rev. D} \textbf{69}, 103523 (2004).

\bibitem{Carroll2001}
S. M. Carroll, ``The Cosmological Constant,'' \textit{Living Rev. Relativ.} \textbf{4}, 1 (2001).

\bibitem{Padmanabhan2005}
T. Padmanabhan, ``Vacuum Fluctuations of Energy Density Can Lead to the Observed Cosmological Constant,'' \textit{Class. Quant. Grav.} \textbf{22}, L107 (2005).

\bibitem{Mori1965}
H. Mori, ``Transport, Collective Motion, and Brownian Motion,'' \textit{Prog. Theor. Phys.} \textbf{33}, 423 (1965).

\bibitem{Zwanzig1960}
R. Zwanzig, ``Ensemble Method in the Theory of Irreversibility,'' \textit{J. Chem. Phys.} \textbf{33}, 1338 (1960).

\bibitem{KemenySnell1960}
J. G. Kemeny and J. L. Snell, \textit{Finite Markov Chains} (Van Nostrand, 1960).

\bibitem{GurvitsLedoux2005}
L. Gurvits and J. Ledoux, ``Markov Property for a Function of a Markov Chain: A Linear Algebra Approach,'' \textit{Lin. Alg. Appl.} \textbf{404}, 85--117 (2005).

\bibitem{Elfving1937}
G. Elfving, ``Zur Theorie der Markoffschen Ketten,'' \textit{Acta Soc. Sci. Fenn. A} \textbf{2}, 1--17 (1937).

\bibitem{Casanellas2023}
M. Casanellas, J. Fern\'{a}ndez-S\'{a}nchez, and J. Roca-Lacostena, ``The Embedding Problem for Markov Matrices,'' \textit{Publ. Mat.} \textbf{67}, 411--445 (2023).

\bibitem{Barandes2025}
J. A. Barandes, ``The Stochastic-Quantum Correspondence,'' \textit{Philosophy of Physics} \textbf{3}, 8 (2025).

\bibitem{Barandes2023}
J. A. Barandes, ``The Stochastic-Quantum Theorem,'' arXiv:2309.03085 (2023).

\bibitem{BarandesHasanKagan2025}
J. A. Barandes, M. Hasan, and D. Kagan, ``The CHSH Game, Tsirelson's Bound, and Causal Locality,'' arXiv:2512.18105 (2025).

\bibitem{Wetterich2025}
C. Wetterich, \textit{The Probabilistic World: Quantum Mechanics from Classical Statistics} (Springer, Fundamental Theories of Physics vol.\ 220, 2025).

\bibitem{Wetterich2009}
C. Wetterich, ``Quantum Mechanics from Classical Statistics,'' \textit{J. Phys.: Conf. Ser.} \textbf{174}, 012008 (2009).

\bibitem{tHooft2016}
G. 't Hooft, \textit{The Cellular Automaton Interpretation of Quantum Mechanics} (Springer, 2016).

\bibitem{Adler2004}
S. L. Adler, \textit{Quantum Theory as an Emergent Phenomenon} (Cambridge University Press, 2004).

\bibitem{Nelson1966}
E. Nelson, ``Derivation of the Schr\"{o}dinger Equation from Newtonian Mechanics,'' \textit{Phys. Rev.} \textbf{150}, 1079 (1966).

\bibitem{Wallstrom1994}
T. C. Wallstrom, ``Inequivalence Between the Schr\"{o}dinger Equation and the Madelung Hydrodynamic Equations,'' \textit{Phys. Rev. A} \textbf{49}, 1613 (1994).

\bibitem{MilzModi2021}
S. Milz and K. Modi, ``Quantum Stochastic Processes and Quantum Non-Markovian Phenomena,'' \textit{PRX Quantum} \textbf{2}, 030201 (2021).

\bibitem{Szangolies2018}
J. Szangolies, ``Epistemic Horizons and the Foundations of Quantum Mechanics,'' \textit{Found. Phys.} \textbf{48}, 1669--1697 (2018).

\bibitem{Sorkin1991}
R. D. Sorkin, ``Spacetime and Causal Sets,'' in \textit{Relativity and Gravitation: Classical and Quantum}, eds. J. C. D'Olivo et al.\ (World Scientific, 1991), pp. 150--173.

\end{thebibliography}

\end{document}
