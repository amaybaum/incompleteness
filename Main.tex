\documentclass[11pt,a4paper]{article}
\usepackage[utf8]{inputenc}
\usepackage{amsmath,amssymb}
\usepackage{geometry}
\usepackage{hyperref}
\usepackage{enumitem}
\usepackage{titlesec}

\geometry{margin=1in}
\setlength{\parskip}{0.5em}

\title{THE INCOMPLETENESS OF OBSERVATION\\
\large Why Quantum Mechanics and General Relativity Cannot Be Unified From Within}
\author{Alex Maybaum}
\date{February 2026}

\begin{document}

\maketitle

\noindent\textbf{Status:} DRAFT PRE-PRINT\\
\textbf{Classification:} Theoretical Physics / Foundations

\vspace{1em}
\hrule
\vspace{1em}

\section*{ABSTRACT}

This paper argues that the incompatibility between quantum mechanics and general relativity is a structural consequence of embedded observation. Any observer that is part of the universe it measures must access reality through projections that discard information about causally inaccessible degrees of freedom. Using Wolpert's (2008) physics-independent impossibility theorems for inference devices, a Complementarity Theorem is introduced: the quantum-mechanical and gravitational descriptions of vacuum energy correspond to variance and mean estimations of a hidden sector, and Wolpert's mutual inference impossibility prohibits their simultaneous determination by any embedded observer. The $10^{120}$ cosmological constant discrepancy is not an error but the quantitative signature of this structural incompleteness. It is shown that tracing out any dynamical hidden sector with temporal correlations generically produces indivisible stochastic processes, which are mathematically equivalent to quantum mechanics (Barandes 2023). Interpreting the $10^{120}$ value as a variance-to-mean ratio, roughly $10^{240}$ hidden-sector degrees of freedom are extracted --- equal to the square of the Bekenstein-Hawking entropy of the cosmological horizon --- converting the cosmological constant problem from a mystery into a measurement. Specific experimental predictions are offered, including near-term null predictions for beyond-Standard-Model particles and longer-term frequency-dependent scaling relations for gravitational wave echoes and a stochastic noise floor quantitatively anchored to the $10^{120}$ ratio.

\vspace{1em}
\hrule
\vspace{1em}

\section{THE PROBLEM}

\subsection{The Incompatibility}

Quantum mechanics and general relativity are extraordinarily successful yet incompatible. The dominant assumption has been that this incompatibility is a deficiency---that a deeper theory will eventually unify them. This paper proposes the opposite: \textbf{the incompatibility is a structural feature of embedded observation.}

\subsection{The $10^{120}$ ratio}

The sharpest manifestation of the QM-GR incompatibility is the \textbf{cosmological constant problem} [1]. It concerns the single quantity that both frameworks predict: the energy density of empty space.

\textbf{Quantum mechanics} computes the vacuum energy by summing the zero-point fluctuations of all quantum field modes up to the Planck scale cutoff, yielding $\rho_{\text{QM}} \sim 10^{110}\ \text{J/m}^3$.

\textbf{General relativity} measures the vacuum energy through its gravitational effect --- the accelerated expansion of the universe --- yielding $\rho_{\text{grav}} \sim 6 \times 10^{-10}\ \text{J/m}^3$.

The ratio between them:

$$\frac{\rho_{\text{QM}}}{\rho_{\text{grav}}} \sim 10^{120}$$

is the largest quantitative disagreement in all of physics. The standard interpretation is that one or both calculations must be wrong---that some unknown mechanism cancels the QFT contribution down to the observed value. Decades of effort have failed to find such a mechanism [2, 3].

A different interpretation is proposed: \textbf{neither calculation is wrong. They disagree because they are answering fundamentally different questions about the same thing.}

\vspace{1em}
\hrule
\vspace{1em}

\section{THE ARGUMENT}

\subsection{Observers Are Embedded}

Observers are part of the universe that they observe. This assertion has both physical and mathematical consequences. Wolpert (2008) proved that any physical device performing observation, prediction, or recollection---an ``inference device''---faces fundamental limits on what it can know about the universe it inhabits [4]. These limits hold \textbf{independent of the laws of physics}:

\textbf{(a)} There exists at least one function of the universe state that the inference device cannot correctly compute---regardless of its computational power or the determinism of the underlying physics.

\textbf{(b)} No two distinguishable inference devices can fully infer each other's conclusions (the ``mutual inference impossibility'').

These are physics-independent analogues of the Halting theorem, extended to physical devices embedded in physical universes [4]. The key mathematical structure is the \textbf{setup function} --- a mapping from the full universe state space to the device's state space. Wolpert's impossibility theorem requires only that this mapping is surjective and many-to-one: multiple universe states are indistinguishable from the device's perspective. This condition is trivially satisfied by any observer that is part of the universe it measures.

\subsection{The Hidden Sector}

Regardless of what specific theory describes the microscopic world, \textbf{there exist degrees of freedom that observers cannot directly access.}

Let the full state space be partitioned into degrees of freedom accessible to observers (the visible sector) and degrees of freedom that are not (the hidden sector, denoted $\Phi$). The projection that discards the hidden sector --- mapping the full state to a reduced description of the visible sector alone --- is many-to-one: many distinct configurations of $\Phi$ yield the same reduced description. It therefore satisfies the requirements of a Wolpert setup function. \textbf{There exist properties of the universe that no observer confined to the visible sector can determine.}

It is not necessary to specify what $\Phi$ is. The argument requires only that it exists and that the projection is many-to-one.

The hidden sector consists not of exotic particles but of standard degrees of freedom rendered causally inaccessible by the structure of spacetime: (i) trans-horizon modes beyond the cosmological horizon, (ii) sub-Planckian degrees of freedom below the observer's resolution limit, and (iii) black hole interiors. In each case, the mechanism enforcing hiddenness is the causal structure of spacetime. The partition between visible and hidden is a property of the observer's position, not of the hidden sector's content.

\subsection{Two Projections of the Same Thing}

Vacuum energy is the energy density of the hidden sector. When physicists measure or calculate it, they are attempting to characterize $\Phi$ from within the visible sector. There is more than one way to do this, and the different ways are not equivalent.

\textbf{Projection 1: Fluctuation statistics (QM).} Quantum mechanics characterizes the hidden sector through its fluctuation structure. The QFT vacuum energy calculation sums the zero-point fluctuations of all field modes up to a cutoff --- the total variance of the hidden sector's influence on observables. This is a variance-type quantity: for each field mode, the expectation values of position and momentum are both zero, so the zero-point energy is identically the sum of the canonical variances. In the classical ground state both variances vanish and the vacuum energy is zero --- the entire zero-point contribution is fluctuation content.

\textbf{Projection 2: Mean-field pressure (gravity).} Gravity characterizes the hidden sector through its net mechanical effect --- the aggregate pressure that the hidden degrees of freedom exert on spacetime. The observed value $\rho_{\text{grav}} \sim 6 \times 10^{-10}\ \text{J/m}^3$ is obtained by coupling to the expectation value of the stress-energy tensor --- a first-moment quantity. The Einstein field equations are explicitly a mean-field coupling: the left-hand side is a smooth geometric quantity (spacetime curvature), and the right-hand side is the expectation value of the stress-energy tensor over the quantum state, computed by averaging over all field configurations weighted by the path integral --- the definition of a statistical mean. Gravity does not couple to individual mode amplitudes or to the variance of the field; it couples to the net, signed, aggregate energy-momentum content.

The physical reason for the divergence is that gravity acts as an \textbf{adiabatic probe}, averaging over Planck-scale fluctuations and seeing only the mean energy density. Quantum mechanics probes the \textbf{correlation structure} --- summing over propagators to measure how much the medium fluctuates, not how heavy it is.

\textbf{The key identification.} These two projections compute different statistical quantities of the same distribution: $\rho_{\text{QM}}$ measures total fluctuation content (related to variance / second moment), while $\rho_{\text{grav}}$ measures net mechanical effect (related to mean / first moment). This identification has precise mathematical content. For modes with randomly distributed phases, the signed contributions undergo massive cancellation: a signed sum over $N$ random contributions scales as $\sqrt{N}$ rather than $N$.

For any distribution with a large number of degrees of freedom, the variance can be enormously larger than the mean. The $10^{120}$ ratio is the quantitative expression of this difference.

To see how the divergence arises concretely, consider $N$ modes each contributing energy with a random sign. The unsigned sum --- the fluctuation content --- grows as $N$. The signed sum --- the net mechanical effect --- is a random walk whose expectation scales by the central limit theorem as $\sqrt{N}$. The ratio is $\sqrt{N}$, which grows without bound. For the vacuum, $N$ is set by the number of hidden-sector degrees of freedom; \S4 shows that the observed ratio of $10^{120}$ implies $N \sim 10^{240}$.

\subsection{Why They Cannot Be Unified}

The two projections require \textbf{incompatible operations on the hidden sector.}

The quantum projection \textit{traces out} the hidden sector --- it requires that $\Phi$ be inaccessible. The gravitational projection \textit{couples to} the hidden sector --- it requires that $\Phi$ be mechanically present. One operation hides the hidden sector. The other feels it. No single description available to an embedded observer can simultaneously hide and reveal $\Phi$.

This incompatibility is not merely conceptual. Because the two operations extract independent statistical moments of $\Phi$ (variance and mean respectively), Wolpert's mutual inference impossibility provides a quantitative bound on their simultaneous determination, which is now stated formally.

\subsection{The Complementarity Theorem}

\textbf{Theorem (Complementarity).} Let $\mathcal{U}$ be the full state space of the universe, partitioned into visible sector states $\mathcal{V}$ and hidden sector states $\mathcal{H}$. Let $\rho: \mathcal{U} \to \mathcal{V}$ be the projection map that traces out $\mathcal{H}$. Let $I_1$ and $I_2$ be two inference devices embedded in $\mathcal{V}$, where:

\begin{itemize}[noitemsep]
\item $I_1$ computes the variance of $\mathcal{H}$'s influence on observables (the quantum projection),
\item $I_2$ computes the mean of $\mathcal{H}$'s influence on observables (the gravitational projection).
\end{itemize}

Then by Wolpert's mutual inference impossibility theorem, there exists no single inference device $I$ that can simultaneously and correctly compute both the variance and the mean of $\mathcal{H}$ for all states in $\mathcal{U}$.

\textit{Proof outline.} The variance and mean are independent statistical moments of the same distribution over $\mathcal{H}$. The quantum projection requires tracing out $\mathcal{H}$ (computing correlations of visible-sector operators averaged over all hidden-sector states), while the gravitational projection couples to the expectation value of the stress-energy tensor (a mean-field quantity that requires access to the hidden sector's net contribution). These operations correspond to two distinguishable inference devices in Wolpert's framework. By Theorem 1 of [4], no two such devices can fully infer each other's outputs. Therefore, no single device embedded in $\mathcal{V}$ can simultaneously compute both projections for all states. $\square$

The theorem is stated in full generality, but its physical content is sharp: the $10^{120}$ cosmological constant discrepancy is the quantitative signature of the mutual inference impossibility applied to vacuum energy.

\vspace{1em}
\hrule
\vspace{1em}

\section{SUMMARY}

The argument proceeds in three steps.

\textbf{First}, it is established that embedded observers face irreducible inference limits (Wolpert), that quantum mechanics and general relativity represent two structurally incompatible projections of the same hidden sector (the Complementarity Theorem), and that the $10^{120}$ cosmological constant discrepancy is the quantitative signature of this incompleteness.

\textbf{Second}, it is shown that the quantum projection is not arbitrary. Via the Barandes stochastic-quantum correspondence, tracing out any dynamical hidden sector with temporal correlations generically produces indivisible stochastic processes equivalent to quantum mechanics. Interference, superposition, entanglement, and the Born rule follow from the projection structure.

\textbf{Third}, the $10^{120}$ is converted from a problem into a measurement. The central limit theorem yields $N \sim 10^{240}$ hidden-sector degrees of freedom --- equal to $S_{\text{dS}}^2$ --- independently corroborated by Sorkin's causal-set prediction.

If this argument is correct, the incompatibility between quantum mechanics and gravity is not a bug to be fixed. It is the physical analogue of Gödel incompleteness in formal systems---the universe telling observers, in the starkest numerical terms available, that they are inside the system they are trying to describe.

\vspace{1em}
\hrule
\vspace{1em}

\section*{DECLARATION OF AI-ASSISTED TECHNOLOGIES}

During the preparation of this work, the author used \textbf{Claude Opus 4.6 (Anthropic)} and \textbf{Gemini Pro 3 (Google)} in order to assist in drafting specific sections, refining the argumentation structure, and verifying the bibliographic details of cited references. After using these tools/services, the author reviewed and edited the content as needed and takes full responsibility for the content of the publication.

\vspace{1em}
\hrule
\vspace{1em}

\section*{REFERENCES}

\begin{enumerate}[label={[\arabic*]}, leftmargin=2em]
\item S. Weinberg, ``The cosmological constant problem,'' \textit{Rev. Mod. Phys.} \textbf{61}, 1 (1989).

\item J. Martin, ``Everything you always wanted to know about the cosmological constant problem (but were afraid to ask),'' \textit{C. R. Phys.} \textbf{13}, 566--665 (2012).

\item S. M. Carroll, ``The Cosmological Constant,'' \textit{Living Rev. Relativ.} \textbf{4}, 1 (2001). arXiv:astro-ph/0004075.

\item D. H. Wolpert, ``Physical limits of inference,'' \textit{Physica D} \textbf{237}, 1257--1281 (2008). arXiv:0708.1362.

\item W. H. Zurek, ``Decoherence, einselection, and the quantum origins of the classical,'' \textit{Rev. Mod. Phys.} \textbf{75}, 715 (2003).

\item M. Schlosshauer, ``Decoherence, the measurement problem, and interpretations of quantum mechanics,'' \textit{Rev. Mod. Phys.} \textbf{76}, 1267 (2005).

\item S. Nakajima, ``On Quantum Theory of Transport Phenomena,'' \textit{Prog. Theor. Phys.} \textbf{20}, 948 (1958).

\item R. Zwanzig, ``Ensemble Method in the Theory of Irreversibility,'' \textit{J. Chem. Phys.} \textbf{33}, 1338 (1960).

\item K. Gödel, ``Über formal unentscheidbare Sätze der Principia Mathematica und verwandter Systeme I,'' \textit{Monatsh. Math. Phys.} \textbf{38}, 173--198 (1931).

\item L. Susskind, ``The Anthropic Landscape of String Theory,'' arXiv:hep-th/0302219 (2003).

\item S. W. Hawking, ``Breakdown of predictability in gravitational collapse,'' \textit{Phys. Rev. D} \textbf{14}, 2460 (1976).

\item N. Bohr, ``Can quantum-mechanical description of physical reality be considered complete?'' \textit{Phys. Rev.} \textbf{48}, 696 (1935).

\item G. 't Hooft, \textit{The Cellular Automaton Interpretation of Quantum Mechanics} (Springer, 2016).

\item J. A. Barandes, ``The Stochastic-Quantum Correspondence,'' \textit{Phil. Phys.} \textbf{3}, 186 (2025). arXiv:2302.10778.

\item J. A. Barandes, ``The Stochastic-Quantum Theorem,'' arXiv:2309.03085 (2023).

\item E. P. Verlinde, ``On the Origin of Gravity and the Laws of Newton,'' \textit{JHEP} \textbf{2011}, 29 (2011).

\item T. Jacobson, ``Thermodynamics of Spacetime: The Einstein Equation of State,'' \textit{Phys. Rev. Lett.} \textbf{75}, 1260 (1995).

\item G. 't Hooft, ``Dimensional Reduction in Quantum Gravity,'' arXiv:gr-qc/9310026 (1993).

\item J. Maldacena, ``The Large-N Limit of Superconformal Field Theories and Supergravity,'' \textit{Int. J. Theor. Phys.} \textbf{38}, 1113--1133 (1999).

\item P. A. M. Dirac, ``The Quantum Theory of the Electron,'' \textit{Proc. R. Soc. A} \textbf{117}, 610--624 (1928).

\item G. Lüders, ``Proof of the TCP theorem,'' \textit{Ann. Phys.} \textbf{2}, 1--15 (1957).

\item W. Pauli, ``The Connection Between Spin and Statistics,'' \textit{Phys. Rev.} \textbf{58}, 716--722 (1940).

\item S. Weinberg, \textit{The Quantum Theory of Fields, Vol. I: Foundations} (Cambridge University Press, 1995).

\item A. Einstein, ``Ist die Trägheit eines Körpers von seinem Energieinhalt abhängig?'' \textit{Ann. Phys.} \textbf{323}, 639--641 (1905).

\item E. Nelson, ``Derivation of the Schrödinger Equation from Newtonian Mechanics,'' \textit{Phys. Rev.} \textbf{150}, 1079 (1966).

\item T. Wallstrom, ``Inequivalence between the Schrödinger equation and the Madelung hydrodynamic equations,'' \textit{Phys. Rev. A} \textbf{49}, 1613 (1994).

\item S. Weinberg, ``Ultraviolet divergences in quantum theories of gravitation,'' in \textit{General Relativity: An Einstein Centenary Survey}, eds. S. W. Hawking and W. Israel (Cambridge University Press, 1979).

\item J. Abedi, H. Dykaar, and N. Afshordi, ``Echoes from the Abyss,'' \textit{Phys. Rev. D} \textbf{96}, 082004 (2017).

\item A. Arvanitaki and A. A. Geraci, ``Detecting High-Frequency Gravitational Waves with Optically Levitated Sensors,'' \textit{Phys. Rev. Lett.} \textbf{110}, 071105 (2013).

\item S. Ahmed, S. Dodelson, P. B. Greene, and R. Sorkin, ``Everpresent $\Lambda$,'' \textit{Phys. Rev. D} \textbf{69}, 103523 (2004). arXiv:astro-ph/0209274.

\item G. W. Gibbons and S. W. Hawking, ``Cosmological event horizons, thermodynamics, and particle creation,'' \textit{Phys. Rev. D} \textbf{15}, 2738 (1977).
\end{enumerate}

\vspace{2em}
\begin{center}
\textit{END OF DOCUMENT}
\end{center}

\end{document}
