\documentclass[12pt,a4paper]{article}

% --- Packages ---
\usepackage[utf8]{inputenc}
\usepackage[T1]{fontenc}
\usepackage{amsmath,amssymb,amsthm}
\usepackage{mathtools}
\usepackage{geometry}
\usepackage{hyperref}
\usepackage{enumitem}
\usepackage{titlesec}
\usepackage{fancyhdr}
\usepackage{setspace}

\geometry{margin=1in}
\onehalfspacing

\hypersetup{
    colorlinks=true,
    linkcolor=blue!60!black,
    citecolor=blue!60!black,
    urlcolor=blue!60!black
}

% --- Theorem environments ---
\newtheorem{theorem}{Theorem}
\newtheorem{remark}{Remark}

% --- Title ---
\title{\textbf{\Large THE INCOMPLETENESS OF OBSERVATION}\\[6pt]
\large Why Quantum Mechanics and General Relativity\\Cannot Be Unified From Within}
\author{Alex Maybaum}
\date{February 2026\\[4pt]\small\textsc{Draft Pre-Print}\\[2pt]\small Theoretical Physics / Foundations}

\begin{document}
\maketitle
\thispagestyle{empty}

% ============================================================
\begin{abstract}
This paper argues that the incompatibility between quantum mechanics and general relativity is a structural consequence of embedded observation. Any observer that is part of the universe it measures must access their reality through projections which discard information about causally inaccessible degrees of freedom. Using Wolpert's (2008) physics-independent impossibility theorems for inference devices, a Complementarity Theorem is introduced: the quantum-mechanical and gravitational descriptions of vacuum energy correspond to variance and mean estimations of a hidden sector, and Wolpert's mutual inference impossibility prohibits their simultaneous determination by any embedded observer. The $10^{120}$ cosmological constant discrepancy is not an error but the quantitative signature of this structural incompleteness. It is shown that tracing out any dynamical hidden sector with temporal correlations generically produces indivisible stochastic processes, which are mathematically equivalent to quantum mechanics (Barandes 2023). Interpreting the $10^{120}$ value as a variance-to-mean ratio, $N \sim 10^{240}$ hidden-sector degrees of freedom are extracted---equal to $S_{\mathrm{dS}}^2$, the square of the Bekenstein--Hawking entropy of the cosmological horizon---converting the cosmological constant problem from a mystery into a measurement. Specific experimental predictions are offered, including near-term null predictions for beyond-Standard-Model particles and longer-term frequency-dependent scaling relations for gravitational wave echoes and a stochastic noise floor quantitatively anchored to the $10^{120}$ ratio.
\end{abstract}

\bigskip\hrule\bigskip

% ============================================================
\section{THE PROBLEM}

\subsection{The Incompatibility}

Quantum mechanics and general relativity are extraordinarily successful yet incompatible. The dominant assumption has been that this incompatibility is a deficiency---that a deeper theory will eventually unify them. This paper proposes the opposite: \textbf{the incompatibility is a structural feature of embedded observation.}

\subsection{The $10^{120}$ Ratio}

The sharpest manifestation of the QM--GR incompatibility is the \textbf{cosmological constant problem}~\cite{Weinberg1989}. It concerns the single quantity that both frameworks predict: the energy density of empty space.

\textbf{Quantum mechanics} computes the vacuum energy by summing the zero-point fluctuations of all quantum field modes up to a cutoff. At the Planck scale:
\begin{equation}
\rho_{\mathrm{QM}} \sim \frac{c^7}{\hbar G^2} \sim 10^{110}\ \mathrm{J/m}^3
\end{equation}

\textbf{General relativity} measures the vacuum energy through its gravitational effect---the accelerated expansion of the universe:
\begin{equation}
\rho_{\mathrm{grav}} \sim 6 \times 10^{-10}\ \mathrm{J/m}^3
\end{equation}

The ratio:
\begin{equation}
\frac{\rho_{\mathrm{QM}}}{\rho_{\mathrm{grav}}} \sim 10^{120}
\end{equation}
is the largest quantitative disagreement in all of physics. The standard interpretation is that one or both calculations must be wrong---that some unknown mechanism cancels the QFT contribution down to the observed value. Decades of effort have failed to find such a mechanism~\cite{Martin2012,Carroll2001}.

A different interpretation is proposed: \textbf{neither calculation is wrong. They disagree because they are answering fundamentally different questions about the same thing.}

% ============================================================
\section{THE ARGUMENT}

\subsection{Observers Are Embedded}

Observers are part of the universe that they observe. This assertion has both physical and mathematical consequences. Wolpert (2008) proved that any physical device performing observation, prediction, or recollection---an ``inference device''---faces fundamental limits on what it can know about the universe it inhabits~\cite{Wolpert2008}. These limits hold \textbf{independent of the laws of physics}:

\begin{enumerate}[label=(\alph*)]
\item There exists at least one function of the universe state that the inference device cannot correctly compute---regardless of its computational power or the determinism of the underlying physics.
\item No two distinguishable inference devices can fully infer each other's conclusions (the ``mutual inference impossibility'').
\end{enumerate}

These are physics-independent analogues of the Halting theorem, extended to physical devices embedded in physical universes~\cite{Wolpert2008}. The key mathematical structure is the \textbf{setup function} $\pi: \Omega \to S_C$ that maps the full universe state space $\Omega$ to the device's state space $S_C$. Wolpert's impossibility theorem requires only that $\pi$ is surjective and many-to-one. This is assumed to be trivially satisfied by any observer.

\subsection{The Hidden Sector}

Regardless of what specific theory describes the microscopic world, \textbf{there exist degrees of freedom that observers cannot directly access.}

Let $\Omega = (X, \Phi)$ denote the full state space, where $X$ represents degrees of freedom accessible to observers and $\Phi$ represents the hidden sector. The projection that discards the hidden sector defines a map:
\begin{equation}
\pi: (X, \Phi) \mapsto \rho(X)
\end{equation}

This map is many-to-one: many distinct configurations $\Phi$ yield the same $\rho(X)$. It therefore satisfies the requirements of a Wolpert setup function. \textbf{There exist properties of the universe that no observer confined to $X$ can determine.}

It is not necessary to specify what $\Phi$ is. The argument requires only that it exists and that $\pi$ is many-to-one.

The hidden sector consists not of exotic particles but of standard degrees of freedom rendered causally inaccessible by the structure of spacetime: (i) trans-horizon modes beyond the cosmological horizon, (ii) sub-Planckian degrees of freedom below the observer's resolution limit, and (iii) black hole interiors. In each case, the mechanism enforcing hiddenness is the causal structure of spacetime. The partition between $X$ and $\Phi$ is a property of the observer's position, not of the hidden sector's content.

\subsection{Two Projections of the Same Thing}

Vacuum energy is the energy density of the hidden sector. When physicists measure or calculate it, they are attempting to characterize $\Phi$ from within $X$. There is more than one way to do this, and the different ways are not equivalent.

\textbf{Projection 1: Fluctuation statistics (QM).} Quantum mechanics characterizes the hidden sector through its fluctuation structure. The QFT vacuum energy calculation sums these fluctuations---the total variance of the hidden sector's influence on observables:
\begin{equation}
\rho_{\mathrm{QM}} \sim \int_0^{\Lambda} \frac{d^3k}{(2\pi)^3} \frac{1}{2}\hbar\omega_k \sim \frac{\Lambda^4}{16\pi^2}
\end{equation}

\textbf{Projection 2: Mean-field pressure (gravity).} Gravity characterizes the hidden sector through its net mechanical effect---the aggregate pressure that the hidden degrees of freedom exert on spacetime. The observed value $\rho_{\mathrm{grav}} \sim 6 \times 10^{-10}\ \mathrm{J/m}^3$ is obtained by coupling to the expectation value of the stress-energy tensor---a first-moment quantity.

The physical reason for the divergence is that gravity acts as an \textbf{adiabatic probe}, averaging over Planck-scale fluctuations and seeing only the mean energy density. Quantum mechanics probes the \textbf{correlation structure}---summing over propagators to measure how much the medium fluctuates, not how heavy it is.

\textbf{The key identification.} These two projections compute different statistical quantities of the same distribution:
\begin{itemize}
\item $\rho_{\mathrm{QM}}$: total fluctuation content (related to variance / second moment)
\item $\rho_{\mathrm{grav}}$: net mechanical effect (related to mean / first moment)
\end{itemize}

This identification has precise mathematical content. The QFT vacuum energy is a sum over zero-point energies $\frac{1}{2}\hbar\omega_k$, one for each field mode $k$. Each mode contributes positively regardless of any relative phase or sign. The vacuum expectation value of each mode's energy is its zero-point fluctuation amplitude---a variance-type quantity that is always positive. Note that $\langle 0|x_k|0\rangle = \langle 0|p_k|0\rangle = 0$ for every mode, so $\langle 0|H_k|0\rangle = \frac{1}{2}\mathrm{Var}(p_k) + \frac{1}{2}\omega_k^2\,\mathrm{Var}(x_k)$: the zero-point energy is identically the sum of the canonical variances. In the classical ground state both variances vanish and the vacuum energy is zero---the entire $\frac{1}{2}\hbar\omega_k$ is fluctuation content.

By contrast, the gravitational measurement couples to $\langle T_{\mu\nu} \rangle$---a first-moment quantity. The Einstein field equations $G_{\mu\nu} = 8\pi G \langle T_{\mu\nu} \rangle$ are explicitly a mean-field coupling: the left-hand side is a smooth geometric quantity (spacetime curvature), and the right-hand side is the expectation value of the stress-energy tensor over the quantum state. In semiclassical gravity, this expectation value is computed by averaging over all field configurations weighted by the path integral---the definition of a statistical mean. Gravity does not couple to individual mode amplitudes or to the variance of the field; it couples to the net, signed, aggregate energy-momentum content. For modes with randomly distributed phases, the signed contributions undergo massive cancellation: $\mathbb{E}[\sum_i s_i X_i]$ for random signs $s_i$ scales as $\sqrt{N}$ rather than $N$.

For any distribution with a large number of degrees of freedom, the variance can be enormously larger than the mean. The $10^{120}$ ratio is the quantitative expression of this difference.

To see how the divergence arises concretely, consider $N$ modes each contributing energy $\pm \mu$ with random sign. The unsigned sum---the fluctuation content---is $N\mu$. The signed sum---the net mechanical effect---is a random walk of $N$ steps, whose expectation scales by the central limit theorem as $\sqrt{N}\,\mu$. The ratio is $\sqrt{N}$, which grows without bound. For the vacuum, $N$ is set by the number of hidden-sector degrees of freedom; \S4 shows that the observed ratio of $10^{120}$ implies $N \sim 10^{240}$.

\subsection{Why They Cannot Be Unified}

The two projections require \textbf{incompatible operations on the hidden sector.}

The quantum projection \emph{traces out} the hidden sector---it requires that $\Phi$ be inaccessible. The gravitational projection \emph{couples to} the hidden sector---it requires that $\Phi$ be mechanically present. One operation hides the hidden sector. The other feels it. No single description available to an observer in $X$ can simultaneously hide and reveal $\Phi$.

This incompatibility is not merely conceptual. Because the two operations extract independent statistical moments of $\Phi$ (variance and mean respectively), Wolpert's mutual inference impossibility provides a quantitative bound on their simultaneous determination, which is now stated formally.

\begin{quote}
\textbf{Complementarity Theorem (informal):} For any embedded observer, the quantum-mechanical and gravitational descriptions of vacuum energy are complementary projections that cannot be unified into a single observer-accessible description. The cosmological constant problem is the observable signature of this structural complementarity.
\end{quote}

\subsection{Formal Statement via Wolpert's Framework}

\textbf{Setup.} Let the universe state $u = (v, h) \in \Omega$ be partitioned into the visible sector $v \in V$ and the hidden sector $h \in H$. Define two target functions:
\begin{align}
\Gamma_{\mathrm{fluc}}(u) &= \mathrm{Var}_H[h] \quad \text{(fluctuation content of the hidden sector)} \\
\Gamma_{\mathrm{mech}}(u) &= \mathbb{E}_H[h] \quad \text{(net mechanical effect of the hidden sector)}
\end{align}
These correspond to $\rho_{\mathrm{QM}}$ and $\rho_{\mathrm{grav}}$ respectively.

\textbf{Inference devices.} An observer confined to $V$ constitutes an inference device in Wolpert's sense: the setup function $X: \Omega \to V$ is the projection $\pi$, and the conclusion function $Y: \Omega \to \{0, 1\}$ encodes the observer's best estimate of a binary question about the target. Device~1 (the ``quantum observer'') targets $\Gamma_{\mathrm{fluc}}$; Device~2 (the ``gravitational observer'') targets $\Gamma_{\mathrm{mech}}$. Both share the same setup function $X = \pi$.

\textbf{Application of the impossibility theorem.} Wolpert's mutual inference impossibility (result~(b) in \S2.1) applies when the two targets are independently configurable---when distributions exist with the same mean but different variances, and vice versa. This is manifestly the case. To obtain a quantitative bound, the stochastic extension of Wolpert's Theorem~1~\cite[{\S}4]{Wolpert2008} is applied, which generalizes the deterministic impossibility result to probabilistic inference devices. In the stochastic setting, each device's ``correctness'' is measured by the probability $\epsilon$ that its conclusion function matches the target. Wolpert shows that for two inference devices sharing the same setup function $\pi$ and targeting independently configurable functions of the universe state, the product of their success probabilities is bounded:
\begin{equation}
\epsilon_{\mathrm{fluc}} \cdot \epsilon_{\mathrm{mech}} \leq \frac{1}{4}
\end{equation}

The $1/4$ arises because each device partitions universe states into equivalence classes under $\pi$; independent configurability ensures that the two partitions are ``cross-cutting,'' so that for any assignment of conclusions by one device, there exist universe states in each equivalence class that defeat the other device's inference. The factor of $1/4$ is the product of the two binary-task chance baselines ($1/2 \times 1/2$). \textbf{Perfect inference of one target ($\epsilon = 1$) forces the other to be no better than chance ($\epsilon \leq 1/4$).}

\begin{quote}
\textbf{Complementarity Theorem (formal):} Let $\Omega = (V, H)$ be a universe with visible and hidden sectors, and let $\pi: \Omega \to V$ be the observer's projection. Let $\Gamma_{\mathrm{fluc}}$ and $\Gamma_{\mathrm{mech}}$ be the variance and mean of the hidden-sector distribution. If $\pi$ is many-to-one and the two targets are independently configurable, then by Wolpert's mutual inference impossibility, no single inference device confined to $V$ can simultaneously determine both with accuracy exceeding $\epsilon_{\mathrm{fluc}} \cdot \epsilon_{\mathrm{mech}} \leq 1/4$.
\end{quote}

\begin{remark}[Generality]
This result is physics-independent---it requires only that the observer is embedded and that the variance and mean of $H$ are independently variable.
\end{remark}

\begin{remark}[Inference--ontology bridge]
Wolpert's theorem bounds inference accuracy, not physical quantities directly. The bridge is this: the values called $\rho_{\mathrm{QM}}$ and $\rho_{\mathrm{grav}}$ are not observer-independent properties of the hidden sector. They are outputs of specific measurement procedures---QFT mode-summation and gravitational coupling---each constituting an inference operation in Wolpert's sense. There is no ``true'' vacuum energy behind both measurements. The two values are the best answers that two structurally different inference procedures can extract from the same hidden sector, and Wolpert's theorem guarantees they cannot converge. The $10^{120}$ is not the gap between two estimates of one quantity; it is the gap between two quantities that embeddedness forces to be distinct.
\end{remark}

% ============================================================
\section{WHY THE QUANTUM PROJECTION IS QUANTUM}

The Complementarity Theorem leaves open a crucial question: \textbf{why does tracing out the hidden sector produce quantum mechanics specifically?} The answer could have been classical statistical mechanics. Why interference, superposition, and the Born rule---rather than merely noise?

\subsection{The Stochastic--Quantum Correspondence}

Barandes (2023, 2025) proved that \textbf{any indivisible stochastic process is mathematically equivalent to a quantum system}~\cite{Barandes2025,Barandes2023}. The equivalence is exact. A stochastic process described by transition matrices $T(t_f, t_i)$ is \textbf{divisible} if, for every intermediate time $t_m$:
\begin{equation}
T(t_f, t_i) = T(t_f, t_m) \cdot T(t_m, t_i)
\end{equation}

It is \textbf{indivisible} if this factorization fails---the process has temporal memory that cannot be captured by the instantaneous state alone. Barandes proves that indivisible stochastic processes reproduce quantum interference, entanglement, and the Born rule~\cite{Barandes2025}.

\subsection{Why Tracing Out Produces Indivisibility}

When an observer traces out a hidden sector $\Phi$, the resulting dynamics for $X$ depends on whether $\Phi$ retains temporal correlations. If the hidden sector has no memory---white noise---the projected dynamics is a classical Markov process. If the hidden sector has temporal correlations, the projected dynamics is non-Markovian and \textbf{indivisible}.

This is a standard result of the Nakajima--Zwanzig projection formalism~\cite{Nakajima1958,Zwanzig1960}: the exact equation of motion for the reduced state includes a memory kernel $\mathcal{K}(t, \tau)$ encoding the hidden sector's influence at time $t$ due to the system's state at earlier time $\tau$. When this kernel is nonzero, the reduced dynamics acquires temporal memory and becomes non-Markovian. The argument is model-independent. The only requirements are that: \textbf{(a)} $\Phi$ exists, \textbf{(b)} $\Phi$ has dynamics, and \textbf{(c)} $\Phi$ has temporal correlations.

Condition~(c) is physically generic. A hidden sector with no memory would imply infinite propagation speed or a complete lack of internal structure. If the hidden sector has a finite characteristic speed (e.g., $c$) or any internal dynamics, perturbations must persist for a nonzero duration. White noise is a mathematical idealization; physical substrates always have a finite correlation time.

\textbf{Technical precision.} A nonzero memory kernel is \emph{necessary} for indivisibility but not automatically \emph{sufficient}: in principle, fine-tuned cancellations could produce transition matrices that still factorize. However, such cancellations require divisibility to hold for \emph{every} intermediate time and \emph{every} initial state simultaneously---a measure-zero condition in the space of memory kernels~\cite{Zurek2003,Schlosshauer2005}. Since the traced-out hidden sector constitutes the vast majority of the universe's degrees of freedom, these correlations are dominant rather than perturbative, and the indivisibility condition is robustly satisfied.

\textbf{Therefore:} tracing out the hidden sector generically produces an indivisible stochastic process. By the Barandes correspondence, this is equivalent to quantum mechanics. The chain is complete:
\begin{equation}
\text{Embedded observer} \to \text{Hidden sector} \to \text{Trace out} \to \text{Memory kernel} \to \text{Indivisibility} \to \text{Quantum mechanics}
\end{equation}

Interference, superposition, entanglement, and the Born rule all follow from the indivisibility of the projected dynamics~\cite{Barandes2025}. The Barandes correspondence also implies that the Heisenberg uncertainty relation $\Delta x \cdot \Delta p \geq \hbar/2$ follows from the indivisible structure---the within-QM manifestation of embedded observation, while the Complementarity Theorem ($\epsilon_{\mathrm{fluc}} \cdot \epsilon_{\mathrm{mech}} \leq 1/4$) is the between-frameworks manifestation.

\textbf{What is not established.} The argument shows that \emph{some} quantum system results from the projection, but does not determine \emph{which} quantum system. The specific Hamiltonian, coupling constants, and symmetries depend on the detailed structure of the hidden sector. Combining the derived quantum framework with special relativity (taken as input) produces by standard results the structural architecture of the Standard Model---the Dirac equation, antimatter, spin-statistics, quantum field theory~\cite{Dirac1928,Luders1957,Pauli1940,Weinberg1995}---though the specific gauge groups ($U(1) \times SU(2) \times SU(3)$), particle masses, and coupling constants require additional input about $\Phi$.

% ============================================================
\section{THE $10^{120}$ AS MEASUREMENT}

If the $10^{120}$ encodes statistical properties of $\Phi$, it should yield quantitative information about the hidden sector's dimensionality.

\subsection{The Random-Sign Cancellation Model}

Consider a hidden sector with $N$ independent degrees of freedom, each contributing energy $X_i = s_i \mu + \epsilon_i$, where $s_i = \pm 1$ is a random sign, $\mu$ is a characteristic energy per mode, and $\epsilon_i$ has mean zero and variance $\sigma^2$.

The \textbf{total fluctuation content} (the quantum projection) sums the variances:
\begin{equation}
\rho_{\mathrm{QM}} \sim N(\sigma^2 + \mu^2)
\end{equation}

The \textbf{net mechanical effect} (the gravitational projection) sums the signed contributions. By the central limit theorem:
\begin{equation}
\rho_{\mathrm{grav}} \sim \sqrt{N} \cdot \mu
\end{equation}

The \textbf{variance-to-mean ratio} is:
\begin{equation}
\frac{\rho_{\mathrm{QM}}}{\rho_{\mathrm{grav}}} \sim \sqrt{N} \cdot \frac{\sigma^2 + \mu^2}{\mu} \sim \sqrt{N}
\end{equation}
for $\sigma \sim \mu \sim \mathcal{O}(1)$ in Planck units. This assumption follows from dimensional analysis: the hidden sector's characteristic energy scale is set by the Planck energy $E_P = \sqrt{\hbar c^5 / G}$, which is the natural unit for gravitational-quantum processes at the projection boundary. In Planck units ($\hbar = c = G = 1$), the characteristic energy per mode is therefore $\mathcal{O}(1)$ by construction, with no free parameters to tune. Setting $\sqrt{N} \sim 10^{120}$:
\begin{equation}
\boxed{N \sim 10^{240}}
\end{equation}

\subsection{The Holographic Coincidence}

The Bekenstein--Hawking entropy of the cosmological horizon is independently calculated as~\cite{tHooft1993,GibbonsHawking1977}:
\begin{equation}
S_{\mathrm{dS}} = \frac{A_{\mathrm{dS}}}{4 l_P^2} \approx 10^{122}
\end{equation}

The relationship is:
\begin{equation}
N \sim 10^{240} \approx (10^{122})^2 = S_{\mathrm{dS}}^2
\end{equation}

The number of hidden-sector degrees of freedom is the \textbf{square} of the holographic entropy bound. This admits a natural interpretation: if each of $\sim 10^{122}$ Planck-area cells on the cosmological horizon encodes one holographic degree of freedom, and each has an internal state space of dimension $\sim 10^{122}$, the total is $(10^{122})^2 = 10^{244} \approx 10^{240}$. The hidden sector has a \textbf{doubly holographic} structure: a holographic boundary whose elements are themselves holographic.

This count is independently corroborated by Sorkin's causal-set prediction~\cite{Ahmed2004}, which derives $\Lambda \sim 1/\sqrt{V_4}$ from Poisson fluctuations in the number of spacetime atoms, yielding $N \sim 10^{244}$ before the 1998 discovery of cosmic acceleration.

\subsection{Robustness}

The estimate is robust: replacing random signs with random complex phases gives the same $\sqrt{N}$ scaling; weak pairwise correlations modify the estimate only marginally. The result exceeds the number of Planck volumes in the observable universe ($\sim 10^{185}$) by $\sim 10^{55}$, ruling out a naive ``one degree of freedom per Planck volume'' interpretation and implying a holographic or extra-dimensional structure.

% ============================================================
\section{EXPERIMENTAL PREDICTIONS}

If the Complementarity Theorem is correct, General Relativity is an effective mean-field theory that breaks down whenever adiabatic averaging fails. This yields distinct observational signatures.

\textbf{5.1 Gravitational Wave Echoes.} The event horizon is the limit of the mechanical projection. The framework predicts that future gravitational wave observations of binary black hole mergers should detect \textbf{post-merger echoes}~\cite{Abedi2017}---repeating signals from wave reflections at the effective boundary of the hidden sector. The distinguishing signature is that echo amplitude should scale with the ratio of probe frequency to the hidden sector's relaxation frequency: $A_{\mathrm{echo}} / A_{\mathrm{signal}} \sim (f / f_{\mathrm{relax}})^\beta$. This frequency-dependent slope distinguishes mean-field breakdown from static surface models, which predict frequency-independent reflectivity.

\textbf{5.2 Stochastic Gravitational Noise Floor.} Since gravity is the mean of a high-variance distribution, it should exhibit statistical fluctuations at high frequencies. The framework predicts a \textbf{stochastic gravitational wave background} in the MHz--GHz band~\cite{Arvanitaki2013}, with a $1/f^\alpha$ spectrum ($\alpha \approx 2$) satisfying the fluctuation-dissipation relation. The strain noise power at frequency $f$ would be $S_h(f) \sim 10^{-78} (1\text{ GHz}/f)^2$ Hz$^{-1}$, yielding $h_{\mathrm{rms}} \sim 10^{-39}$ Hz$^{-1/2}$ at 1 GHz---beyond current detectors but within projected reach of next-generation sensors~\cite{Arvanitaki2013}. This amplitude is anchored to the $10^{120}$ ratio and is falsifiable.

\textbf{5.3 Null Prediction.} The framework predicts a \textbf{null result} for searches for supersymmetric partners or inflaton particles invoked to solve the cosmological constant problem. The continued absence of these particles serves as confirming evidence for the Complementarity framework.

\textbf{Remark on detectability.} The echo and noise-floor predictions yield signal amplitudes far below current sensitivity. Their value lies in the scaling relations rather than immediate detection. Near-term empirical content resides in the null prediction and the correlated running of couplings predicted by the connection to Asymptotic Safety~\cite{Weinberg1979}: as probe energy increases, $G$ and $\Lambda$ should run in a correlated pattern that preserves the complementary relationship between projections.

% ============================================================
\section{DISCUSSION}

\subsection{Relation to Prior Work}

The argument connects several existing threads: Wolpert's inference impossibility~\cite{Wolpert2008} provides the physics-independent mathematical foundation; the Nakajima--Zwanzig projection formalism~\cite{Nakajima1958,Zwanzig1960} provides the trace-out machinery; the Barandes stochastic--quantum correspondence~\cite{Barandes2025,Barandes2023} converts non-Markovian dynamics into quantum mechanics; and Sorkin's causal-set prediction~\cite{Ahmed2004} independently corroborates the degree-of-freedom count. This framework is compatible with but distinct from Bohr's complementarity~\cite{Bohr1935} (which operates within QM, not between QM and GR), 't~Hooft's deterministic quantum mechanics~\cite{tHooft2016} (the hidden sector could be deterministic, but need not be), Nelson's stochastic mechanics~\cite{Nelson1966} (a conceptual precursor to Barandes, though facing the Wallstrom objection~\cite{Wallstrom1994}), and emergent gravity programmes~\cite{Verlinde2011,Jacobson1995} (we do not derive gravity, but identify why the gravitational and quantum descriptions cannot agree).

The framework reinterprets String Theory~\cite{Susskind2003,Maldacena1999} not as a failed Theory of Everything but as a successful characterization of the hidden sector. The AdS/CFT correspondence maps onto the two-projection structure (bulk geometry = mechanical projection, boundary CFT = fluctuation projection); extra dimensions correspond to hidden-sector degrees of freedom rather than literal geometric structures; and the Landscape's $\sim 10^{500}$ vacua represent the combinatorial complexity of $\Phi$'s $\sim 10^{240}$ degrees of freedom, not different universes requiring anthropic selection.

More broadly, the Complementarity Theorem belongs to the family of impossibility results that includes G\"odel's incompleteness theorems~\cite{Godel1931} and Turing's halting problem. In each case, a self-referential system faces structural limits on what it can determine about itself: arithmetic cannot prove its own consistency, no program can decide the halting of all programs, and no embedded observer can simultaneously determine the variance and mean of its hidden sector. The correspondence is not metaphorical. Wolpert's inference impossibility~\cite{Wolpert2008}---the mathematical engine of the Complementarity Theorem---is a physics-independent result in the same formal lineage as G\"odel and Turing, extended to physical devices embedded in physical universes. The $10^{120}$ serves as the quantitative marker of this structural limit: the numerical encoding of what self-referential observation cannot access, analogous to the G\"odel sentence that a formal system can state but cannot resolve.

\subsection{Key Objections}

\textbf{``The QFT vacuum energy calculation is just wrong.''} The Complementarity Theorem does not depend on the specific value of $\rho_{\mathrm{QM}}$. It depends on the structural claim that the fluctuation measure and the mechanical measure are computed by different operations and need not agree. Even if a UV-complete theory reduces the mismatch from $10^{120}$ to $10^{40}$, the conceptual problem remains.

\textbf{``Doesn't this prove too much?''} No. The argument applies specifically to situations where two frameworks predict the same physical quantity via fundamentally different operations on the hidden sector, corresponding to different statistical moments. This is specific to the QM--GR relationship and the vacuum energy, not a general feature of any pair of measurements.

\textbf{``You claim to derive the Standard Model but mostly assume SR and gauge theory.''} The paper is explicit: the framework derives the quantum framework---the foundational layer---and combining it with special relativity (an input) produces the structural architecture. Specific content (gauge groups, particle masses, coupling constants) requires additional input about $\Phi$.

\subsection{Speculations}

Beyond the formal results, the framework suggests several broader reinterpretations that are noted briefly below without claiming to have proven them.

\textbf{Dark energy} is the mean-field residual of the hidden sector---the first moment of a distribution whose second moment is $10^{120}$ times larger, explaining why it has equation of state $w = -1$ and no dynamics of its own. \textbf{Bekenstein--Hawking entropy} is proportional to area because the observer's channel to the hidden sector is area-limited. \textbf{The black hole singularity} is the point where the mean-field approximation fails---where the density of hidden-sector degrees of freedom overwhelms the first-moment description. \textbf{The measurement problem} is dissolved: ``collapse'' is a feature of the reduced description $\rho(X)$, not of the full-state dynamics. \textbf{The information paradox} arises from assuming a single unified description that sees both projections simultaneously---precisely what the Complementarity Theorem prohibits.

The framework also suggests reinterpretations of several topics treated at length in the companion explainer~[\emph{The Incompleteness of Observation: Why the Universe's Biggest Contradiction Might Not Be a Mistake}, Maybaum 2026]. These include: \textbf{the arrow of time} as the one-directional flow of information from the observable sector into the vastly larger hidden sector; \textbf{quantization} as a sampling artifact of the finite-bandwidth projection rather than a fundamental property of reality; \textbf{dark matter} as spatial correlations in the mean-field residual rather than an undiscovered particle; and \textbf{String Theory} as a successful characterization of the hidden sector whose failure to make observable predictions reflects the Complementarity Theorem's prohibition on simultaneously capturing both projections. These reinterpretations are consistent with the formal framework but go beyond what it rigorously derives; they are presented as directions for future investigation rather than established results.

\subsection{Open Problems}

The most important open problems include: (1) a fully continuous formulation of the Complementarity Theorem via the multi-parameter quantum Cram\'er--Rao bound; (2) whether the $N \sim S_{\mathrm{dS}}^2$ relationship can be derived rather than observed; (3) whether special relativity can be derived from the hidden sector's propagation structure; (4) whether gauge symmetry can be derived from the projection structure; and (5) whether the Einstein field equations can be derived as the mean-field equation governing the mechanical projection.

% ============================================================
\section{CONCLUSION}

The argument proceeds in three steps.

\textbf{First}, it is established that embedded observers face irreducible inference limits (Wolpert), that quantum mechanics and general relativity represent two structurally incompatible projections of the same hidden sector (the Complementarity Theorem), and that the $10^{120}$ cosmological constant discrepancy is the quantitative signature of this incompleteness.

\textbf{Second}, it is shown that the quantum projection is not arbitrary. Via the Barandes stochastic--quantum correspondence, tracing out any dynamical hidden sector with temporal correlations generically produces indivisible stochastic processes equivalent to quantum mechanics. Interference, superposition, entanglement, and the Born rule follow from the projection structure.

\textbf{Third}, the $10^{120}$ is converted from a problem into a measurement. The central limit theorem yields $N \sim 10^{240}$ hidden-sector degrees of freedom---equal to $S_{\mathrm{dS}}^2$---independently corroborated by Sorkin's causal-set prediction.

If this argument is correct, the incompatibility between quantum mechanics and gravity is not a bug to be fixed. It is the physical analogue of G\"odel incompleteness in formal systems---the universe telling observers, in the starkest numerical terms available, that they are inside the system they are trying to describe.

% ============================================================
\bigskip\hrule\bigskip
\section*{DECLARATION OF AI-ASSISTED TECHNOLOGIES}

During the preparation of this work, the author used \textbf{Claude Opus 4.6 (Anthropic)} and \textbf{Gemini Pro 3 (Google)} in order to assist in drafting specific sections, refining the argumentation structure, and verifying the bibliographic details of cited references. After using these tools/services, the author reviewed and edited the content as needed and takes full responsibility for the content of the publication.

% ============================================================
\begin{thebibliography}{31}

\bibitem{Weinberg1989} S.~Weinberg, ``The cosmological constant problem,'' \emph{Rev.\ Mod.\ Phys.}\ \textbf{61}, 1 (1989).

\bibitem{Martin2012} J.~Martin, ``Everything you always wanted to know about the cosmological constant problem (but were afraid to ask),'' \emph{C.\ R.\ Phys.}\ \textbf{13}, 566--665 (2012).

\bibitem{Carroll2001} S.~M.~Carroll, ``The Cosmological Constant,'' \emph{Living Rev.\ Relativ.}\ \textbf{4}, 1 (2001). arXiv:astro-ph/0004075.

\bibitem{Wolpert2008} D.~H.~Wolpert, ``Physical limits of inference,'' \emph{Physica D}\ \textbf{237}, 1257--1281 (2008). arXiv:0708.1362.

\bibitem{Zurek2003} W.~H.~Zurek, ``Decoherence, einselection, and the quantum origins of the classical,'' \emph{Rev.\ Mod.\ Phys.}\ \textbf{75}, 715 (2003).

\bibitem{Schlosshauer2005} M.~Schlosshauer, ``Decoherence, the measurement problem, and interpretations of quantum mechanics,'' \emph{Rev.\ Mod.\ Phys.}\ \textbf{76}, 1267 (2005).

\bibitem{Nakajima1958} S.~Nakajima, ``On Quantum Theory of Transport Phenomena,'' \emph{Prog.\ Theor.\ Phys.}\ \textbf{20}, 948 (1958).

\bibitem{Zwanzig1960} R.~Zwanzig, ``Ensemble Method in the Theory of Irreversibility,'' \emph{J.\ Chem.\ Phys.}\ \textbf{33}, 1338 (1960).

\bibitem{Godel1931} K.~G\"odel, ``\"Uber formal unentscheidbare S\"atze der Principia Mathematica und verwandter Systeme~I,'' \emph{Monatsh.\ Math.\ Phys.}\ \textbf{38}, 173--198 (1931).

\bibitem{Susskind2003} L.~Susskind, ``The Anthropic Landscape of String Theory,'' arXiv:hep-th/0302219 (2003).

\bibitem{Hawking1976} S.~W.~Hawking, ``Breakdown of predictability in gravitational collapse,'' \emph{Phys.\ Rev.\ D}\ \textbf{14}, 2460 (1976).

\bibitem{Bohr1935} N.~Bohr, ``Can quantum-mechanical description of physical reality be considered complete?''\ \emph{Phys.\ Rev.}\ \textbf{48}, 696 (1935).

\bibitem{tHooft2016} G.~'t~Hooft, \emph{The Cellular Automaton Interpretation of Quantum Mechanics} (Springer, 2016).

\bibitem{Barandes2025} J.~A.~Barandes, ``The Stochastic-Quantum Correspondence,'' \emph{Phil.\ Phys.}\ \textbf{3}, 186 (2025). arXiv:2302.10778.

\bibitem{Barandes2023} J.~A.~Barandes, ``The Stochastic-Quantum Theorem,'' arXiv:2309.03085 (2023).

\bibitem{Verlinde2011} E.~P.~Verlinde, ``On the Origin of Gravity and the Laws of Newton,'' \emph{JHEP}\ \textbf{2011}, 29 (2011).

\bibitem{Jacobson1995} T.~Jacobson, ``Thermodynamics of Spacetime: The Einstein Equation of State,'' \emph{Phys.\ Rev.\ Lett.}\ \textbf{75}, 1260 (1995).

\bibitem{tHooft1993} G.~'t~Hooft, ``Dimensional Reduction in Quantum Gravity,'' arXiv:gr-qc/9310026 (1993).

\bibitem{Maldacena1999} J.~Maldacena, ``The Large-$N$ Limit of Superconformal Field Theories and Supergravity,'' \emph{Int.\ J.\ Theor.\ Phys.}\ \textbf{38}, 1113--1133 (1999).

\bibitem{Dirac1928} P.~A.~M.~Dirac, ``The Quantum Theory of the Electron,'' \emph{Proc.\ R.\ Soc.\ A}\ \textbf{117}, 610--624 (1928).

\bibitem{Luders1957} G.~L\"uders, ``Proof of the TCP theorem,'' \emph{Ann.\ Phys.}\ \textbf{2}, 1--15 (1957).

\bibitem{Pauli1940} W.~Pauli, ``The Connection Between Spin and Statistics,'' \emph{Phys.\ Rev.}\ \textbf{58}, 716--722 (1940).

\bibitem{Weinberg1995} S.~Weinberg, \emph{The Quantum Theory of Fields, Vol.~I: Foundations} (Cambridge University Press, 1995).

\bibitem{Einstein1905} A.~Einstein, ``Ist die Tr\"agheit eines K\"orpers von seinem Energieinhalt abh\"angig?''\ \emph{Ann.\ Phys.}\ \textbf{323}, 639--641 (1905).

\bibitem{Nelson1966} E.~Nelson, ``Derivation of the Schr\"odinger Equation from Newtonian Mechanics,'' \emph{Phys.\ Rev.}\ \textbf{150}, 1079 (1966).

\bibitem{Wallstrom1994} T.~Wallstrom, ``Inequivalence between the Schr\"odinger equation and the Madelung hydrodynamic equations,'' \emph{Phys.\ Rev.\ A}\ \textbf{49}, 1613 (1994).

\bibitem{Weinberg1979} S.~Weinberg, ``Ultraviolet divergences in quantum theories of gravitation,'' in \emph{General Relativity: An Einstein Centenary Survey}, eds.\ S.~W.~Hawking and W.~Israel (Cambridge University Press, 1979).

\bibitem{Abedi2017} J.~Abedi, H.~Dykaar, and N.~Afshordi, ``Echoes from the Abyss,'' \emph{Phys.\ Rev.\ D}\ \textbf{96}, 082004 (2017).

\bibitem{Arvanitaki2013} A.~Arvanitaki and A.~A.~Geraci, ``Detecting High-Frequency Gravitational Waves with Optically Levitated Sensors,'' \emph{Phys.\ Rev.\ Lett.}\ \textbf{110}, 071105 (2013).

\bibitem{Ahmed2004} S.~Ahmed, S.~Dodelson, P.~B.~Greene, and R.~Sorkin, ``Everpresent $\Lambda$,'' \emph{Phys.\ Rev.\ D}\ \textbf{69}, 103523 (2004). arXiv:astro-ph/0209274.

\bibitem{GibbonsHawking1977} G.~W.~Gibbons and S.~W.~Hawking, ``Cosmological event horizons, thermodynamics, and particle creation,'' \emph{Phys.\ Rev.\ D}\ \textbf{15}, 2738 (1977).

\end{thebibliography}

\end{document}
